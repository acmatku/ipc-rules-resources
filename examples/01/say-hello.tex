\documentclass[a4paper]{article}

\usepackage[english]{babel}
\usepackage[utf8]{inputenc}
\usepackage{amsmath}
\usepackage{graphicx}
\usepackage{listings}
\usepackage[parfill]{parskip}
\usepackage[colorinlistoftodos]{todonotes}
\usepackage[colorlinks=true,urlcolor=blue,linkcolor=blue]{hyperref}

\title{Example Problem 01: Say Hello!}
\date{} % date intentionally omitted

\begin{document}

\lstset{
    language=java,
    basicstyle=\ttfamily,
    numbers=left,
    numbersep=8pt,
    showspaces=false,
    showstringspaces=false
}
\maketitle
\textbf{Description:} Write a program to send a friendly greeting! 

\textbf{Input:} The first line will be an integer representing the number of lines to read. Each subsequent line will be a single name.

\textbf{Output:} On a new line for each, print out the name you read as part of the string ``Hello \textless name\textgreater!''.

\textbf{Sample:}

\begin{tabular}{|p{0.47\textwidth}|p{0.47\textwidth}|}
    \hline
    \textbf{input} & \textbf{outout} \\
    \hline
    \begin{verbatim}
    3
    World
    Alan
    Ada
    \end{verbatim} &
    \begin{verbatim}
    Hello world!
    Hello Alan!
    Hello Ada!
    \end{verbatim} \\
    \hline
\end{tabular}

\end{document}
