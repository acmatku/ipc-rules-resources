\documentclass[a4paper]{article}

\usepackage[english]{babel}
\usepackage[utf8]{inputenc}
\usepackage{amsmath}
\usepackage{graphicx}
\usepackage{listings}
\usepackage[parfill]{parskip}
\usepackage[colorinlistoftodos]{todonotes}
\usepackage[colorlinks=true,urlcolor=blue,linkcolor=blue]{hyperref}

\title{Example Problem 2: Fractions}
\date{} % date intentionally omitted

\begin{document}

\lstset{
    language=java,
    basicstyle=\ttfamily,
    numbers=left,
    numbersep=8pt,
    showspaces=false,
    showstringspaces=false
}
\maketitle
\textbf{Description:} Write a program to simplify fractions.

\textbf{Input:} Write your program to read pairs of integers from \texttt{stdin}. On each line of input, there will be a pair of space separated integers. The first will represent the \texttt{numerator} of the fraction, the second will represent the \texttt{denominator}. The end of input will be signaled by a \texttt{0}. For a pair of integers, you will never be given a numerator or denominator with a value of \texttt{0}. The only \texttt{0} given as input will be the one that signals the end of input. 


\textbf{Output:} Write to \texttt{stdout} the simplified fraction, with the numerator separated from the denominator by a \textit{forward slash}, \texttt{/}. Each simplified fraction should be separated from the next by a newline. If the simplified fraction would be a single integer, write it without a denominator.  

\textbf{Sample:}

\begin{tabular}{|p{0.47\textwidth}|p{0.47\textwidth}|}
    \hline
    \textbf{input} & \textbf{outout} \\
    \hline
    \begin{verbatim}
    18 12
    28 49
    99 11
    13 5
    0
    \end{verbatim} &
    \begin{verbatim}
    3/2
    4/7
    9
    13/5
    \end{verbatim} \\
    \hline
\end{tabular}
\end{document}
